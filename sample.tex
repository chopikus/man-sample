\documentclass[a4paper, 14pt]{article}
\usepackage{fontspec}
\setmainfont[Ligatures=TeX]{Times New Roman}
\usepackage[russian, ukrainian, english]{babel}
\usepackage{extsizes}
\usepackage{geometry}
\usepackage{setspace}
\usepackage{tocloft}
 \geometry{
 a4paper,
 total={170mm,257mm},
 left=2cm,
 top=2cm,
 bottom=2cm,
 right=2cm	
 }
\def\changemargin#1#2{\list{}{\rightmargin#2\leftmargin#1}\item[]}
\let\endchangemargin=\endlist 
\pagestyle{myheadings}
\begin{document}
\begin{titlepage}
\begin{center}
Міністерство освіти і науки України\\
Департамент науки і освіти Харківської облдержадміністрації\\
Харківське територіальне відділення МАН України\\
\end{center}
\par\null\par
\begin{changemargin}{10cm}{0cm}
Відділення: комп'ютерних наук\\
Секція: комп'ютерні системи та мережі
\end{changemargin}
\par\null\par
\begin{center}
РОЗРОБКА МЕТОДА ВИПРАВЛЕННЯ, ЯК НЕЗАЛЕЖНИХ, ТАК І ГРУПОВИХ
ПОМИЛОК, ЩО ВИНИКАЮТЬ ПРИ ОБМІНІ ІНФОРМАЦІЄЮ В
КОМП'ЮТЕРНИХ МЕРЕЖАХ
\end{center}
\par\null\par\null\par\null
\begin{changemargin}{10cm}{0cm}
Роботу виконав:\\
Волков Юрій Володимирович,\\
учень 10 класу Харківського\\
Навчально-виховного комплексу\\
№45 «Академічна гімназія»\\
Харківської міської ради\\
Харківської області
\end{changemargin}
\par
\begin{changemargin}{10cm}{0cm}
Науковий керівник:\\
Руккас Кирило Маркович,\\
професор кафедри теоретичної та\\
прикладної інформатики\\
механіко-математичного\\
факультету Харківського\\
національного університету\\
ім. В.Н. Каразіна, доктор\\
технічних наук, доцент\\
\vspace*{\fill}
\end{changemargin}
\begin{center}
Харків -- 2014 
\end{center}
\end{titlepage}
\setstretch{1.3}
\setcounter{page}{2}
Nam congue neque eu molestie varius. Curabitur gravida lobortis orci id dignissim. Mauris eget dui leo. Suspendisse a sem ac sapien lobortis eleifend non viverra neque. Fusce laoreet elit eu ornare malesuada. Etiam ut tincidunt risus, id eleifend ipsum. Cras laoreet est id vestibulum gravida. Ut malesuada arcu diam, ut convallis quam rhoncus sed. Donec ut augue viverra, commodo urna at, sagittis arcu.
\end{document}
