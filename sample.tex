\documentclass[a4paper, 14pt]{article}
\usepackage{fontspec, extsizes, geometry, titlesec, fancyhdr, titletoc, graphicx, float, setspace, caption}
\usepackage[russian, english, ukrainian]{babel}
\setmainfont[Ligatures=TeX]{Times New Roman}
\geometry{
 a4paper,
 total={170mm,257mm},
 left=2cm,
 top=2cm,
 bottom=2cm,
 right=2cm	
} %отступы
\def\numberline#1{#1. } % фикс содержания
%команда для удобных отступов в титулке
\def\changemargin#1#2{\list{}{\rightmargin#2\leftmargin#1}\item[]}
\let\endchangemargin=\endlist 
%
%новая секция -> новая страница
\let\stdsection\section
\renewcommand\section{\newpage\stdsection}
%
\addto\captionsenglish{\renewcommand{\figurename}{Рис.}} %Типо не малюнок а РИСУНОК. Так и стоит в нормах
\titleformat{\section}[display]{\filcenter}{РОЗДІЛ \bfseries\thesection. }{0pt}{\bfseries\MakeUppercase} %Ставим название раздела по середине, делаем его КАПСЛОКОМ и переносим его на следующую строку (а также делаем жирным, спасибо @S1eap (#2))
\renewcommand{\thesubsection}{\arabic{section}.\arabic{subsection}} %чиним номер подраздела
\titleformat{\subsection}{\filcenter}{\bfseries \thesubsection. }{0pt}{\bfseries}{} %чиним название подраздела (а также центрируем и делаем жирным, спасибо @S1eap (#2))
\captionsetup{labelsep=period} % "Рис. 1." вместо "Рис. 1:" (спасибо @S1eap)
\makeatletter
\renewcommand*\l@section{\@dottedtocline{1}{1.5em}{2.3em}}
\makeatother %точечки в содержании
%начиная отсюда сложная штука для того, чтобы номер страницы был справа сверху на всех страничках
\fancyhf{}
\renewcommand{\headrulewidth}{0pt}
\newcommand{\changefont}{
  \fontsize{14}{14}\selectfont
}
\fancyhead[R]{\changefont \thepage}
\fancypagestyle{plain}{
\fancyhf{}
\fancyhead[R]{\changefont \thepage}
\renewcommand{\headrulewidth}{0pt}
\renewcommand{\footrulewidth}{0pt}
}
\pagestyle{fancy}
%конец сложной штуки
\linespread{1.25} % #1 спасибо @kolbaser



\begin{document}
% НАЧАЛО ТИТУЛКИ
\thispagestyle{empty} %чтобы номер страницы в титулке не показывался
\begin{spacing}{1}
\begin{center}
Міністерство освіти і науки України\\ % 2 бэкслеша - новая строка
Департамент науки і освіти Харківської облдержадміністрації\\
Харківське територіальне відділення МАН України\\
\end{center}
\par\null\par % расстояние в 2 интервала
\begin{changemargin}{10cm}{0cm}
Відділення: комп'ютерних наук\\
Секція: комп'ютерні системи та мережі
\end{changemargin}
\par\null\par % расстояние в 2 интервала
\begin{center}
РОЗРОБКА МЕТОДА ВИПРАВЛЕННЯ, ЯК НЕЗАЛЕЖНИХ, ТАК І ГРУПОВИХ
ПОМИЛОК, ЩО ВИНИКАЮТЬ ПРИ ОБМІНІ ІНФОРМАЦІЄЮ В
КОМП'ЮТЕРНИХ МЕРЕЖАХ
\end{center}
\par\null\par\null\par\null %расстояние в 3 интервала
\begin{changemargin}{10cm}{0cm}
Роботу виконав:\\ 
Волков Юрій Володимирович,\\
учень 10 класу Харківського\\
Навчально-виховного комплексу\\
№45 «Академічна гімназія»\\
Харківської міської ради\\
Харківської області
\end{changemargin}
\par % расстояние в 1 интервал
\begin{changemargin}{10cm}{0cm}
Науковий керівник:\\
Руккас Кирило Маркович,\\
професор кафедри теоретичної та\\
прикладної інформатики\\
механіко-математичного\\
факультету Харківського\\
національного університету\\
ім. В.Н. Каразіна, доктор\\
технічних наук, доцент\\
\vspace*{\fill}
\end{changemargin}
\begin{center}
Харків -- 2014 
\end{center}
\end{spacing}
% КОНЕЦ ТИТУЛКИ

\section*{Т\lowercase{ези}}
Автор работы: анон
...
\newpage

\renewcommand{\contentsname}{\normalsize \normalfont ЗМІСТ} %меняем название странички с содержанием, добавлено \normalfont (спасибо @mojicfg)

\tableofcontents %генерация содержания

\newpage
\addcontentsline{toc}{section}{ВСТУП} %добавляем страницу ВСТУП в содержание
\section*{\textbf{ВСТУП}}
Ваш ВСТУП
\newpage %ну и дальше Ваш ман
\section{Основна частина}
Lorem Ipsum - это текст-"рыба", часто используемый в печати и вэб-дизайне. Lorem Ipsum является стандартной "рыбой" для текстов на латинице с начала XVI века. В то время некий безымянный печатник создал большую коллекцию размеров и форм шрифтов, используя Lorem Ipsum для распечатки образцов. Lorem Ipsum не только успешно пережил без заметных изменений пять веков, но и перешагнул в электронный дизайн. Его популяризации в новое время послужили публикация листов Letraset с образцами Lorem Ipsum в 60-х годах и, в более недавнее время, программы электронной вёрстки типа Aldus PageMaker, в шаблонах которых используется Lorem Ipsum.
\begin{figure}[h]
    \centering
    \includegraphics[width=0.25\textwidth]{example-image-a}
    \caption{Правильно подпись делает, между прочим!}
    \label{fig:mesh1}
\end{figure}
 
\subsection{Подраздел}
\subsubsection{Подподраздел}
\section{очень-очень-очень-очень-очень-очень-очень-очень длинное название раздела (такое бывает)}
\addcontentsline{toc}{section}{СПИСОК ВИКОРИСТАНИХ ДЖЕРЕЛ} 
\section*{список використаних джерел}
Ваш список источников
\end{document}

