\documentclass[a4paper, 14pt]{article}
\usepackage{fontspec, extsizes, geometry, setspace, titlesec, fancyhdr, graphicx, float, setspace, caption}
\usepackage[russian, english, ukrainian]{babel}
\usepackage[dotinlabels]{titletoc}
\setmainfont[Ligatures=TeX]{Times New Roman}
\geometry{a4paper,total={170mm,257mm},left=2cm,top=2cm,bottom=2cm,right=2cm}
\def\changemargin#1#2{\list{}{\rightmargin#2\leftmargin#1}\item[]}\let\endchangemargin=\endlist % удобная команда
\makeatletter\newcommand\Dotfill{\leavevmode\leaders\hb@xt@0.65em{\hss.\hss}\hfill}\makeatother % удобная команда
\let\stdsection\section\renewcommand\section{\newpage\stdsection} % Новая секция -> новая страница
\addto\captionsenglish{\renewcommand{\figurename}{Рис.}} % Подпись картинок
\titleformat{\section}[display]{\filcenter}{\bfseries{РОЗДІЛ \thesection.}}{0pt}{\bfseries\MakeUppercase} % Изменение заголовка всех разделов
\renewcommand{\thesubsection}{\arabic{section}.\arabic{subsection}} % Чиним номер подраздела
\titleformat{\subsection}{\filcenter}{\bfseries \thesubsection. }{0pt}{\bfseries}{} %чиним название подраздела 
\captionsetup{labelsep=period} % "Рис. 1." вместо "Рис. 1:"
\fancyhf{}\renewcommand{\headrulewidth}{0pt}\newcommand{\changefont}{\fontsize{14}{14}\selectfont}\fancyhead[R]{\changefont \thepage}\fancypagestyle{plain}{\fancyhf{}\fancyhead[R]{\changefont \thepage}\renewcommand{\headrulewidth}{0pt}\renewcommand{\footrulewidth}{0pt}}\pagestyle{fancy} %номер страницы справа сверху на всех страничках [это ужас]
\linespread{1.25} %#1
\renewcommand{\contentsname}{ЗМІСТ} %изменяем название странички с содержанием
\def\numberline#1{#1. } % Фикс чтобы названия не налезали друг на друга в содержании
\titlecontents{section}[0pt]{\normalfont}{РОЗДІЛ \thecontentslabel. }{}{\Dotfill \contentspage} % оформление разделов, точек в содержании
\titlecontents{subsection}[15pt]{\normalfont}{\thecontentslabel. }{}{\Dotfill \contentspage} % оформление подразделов, точек в содержании
\titlecontents{subsubsection}[30pt]{\normalfont}{\thecontentslabel. }{}{\Dotfill \contentspage} % оформление подподразделов, точек в содержании

\begin{document}
% Титулка
\thispagestyle{empty}
\begin{spacing}{1}
\begin{center}
Міністерство освіти і науки України\\
Департамент науки і освіти Харківської облдержадміністрації\\
Харківське територіальне відділення МАН України\\
\end{center}\par\null\par
\begin{changemargin}{10cm}{0cm}
Відділення: комп'ютерних наук\\
Секція: комп'ютерні системи та мережі
\end{changemargin}\par\null\par
\begin{center}
РОЗРОБКА МЕТОДА ВИПРАВЛЕННЯ, ЯК НЕЗАЛЕЖНИХ, ТАК І ГРУПОВИХ
ПОМИЛОК, ЩО ВИНИКАЮТЬ ПРИ ОБМІНІ ІНФОРМАЦІЄЮ В
КОМП'ЮТЕРНИХ МЕРЕЖАХ
\end{center}\par\null\par\null\par\null
\begin{changemargin}{10cm}{0cm}
Роботу виконав:\\ 
Волков Юрій Володимирович,\\
учень 10 класу Харківського\\
Навчально-виховного комплексу\\
№45 «Академічна гімназія»\\
Харківської міської ради\\
Харківської області
\end{changemargin}\par
\begin{changemargin}{10cm}{0cm}
Науковий керівник:\\
Руккас Кирило Маркович,\\
професор кафедри теоретичної та\\
прикладної інформатики\\
механіко-математичного\\
факультету Харківського\\
національного університету\\
ім. В.Н. Каразіна, доктор\\
технічних наук, доцент\\
\vspace*{\fill}\end{changemargin}
\begin{center}
Харків -- \the\year{}
\end{center}\end{spacing}

% Тези
\section*{Т\lowercase{ези}}
Автор работы: анон
...
\newpage
\tableofcontents %генерация содержания
\newpage
\section*{\textbf{ВСТУП}}
\addcontentsline{toc}{section}{ВСТУП} %добавляем страницу ВСТУП в содержание
Ваш ВСТУП
\newpage %ну и дальше Ваш ман
\section{Основна частина}
Lorem Ipsum - это текст-"рыба", часто используемый в печати и вэб-дизайне. Lorem Ipsum является стандартной "рыбой" для текстов на латинице с начала XVI века. В то время некий безымянный печатник создал большую коллекцию размеров и форм шрифтов, используя Lorem Ipsum для распечатки образцов. Lorem Ipsum не только успешно пережил без заметных изменений пять веков, но и перешагнул в электронный дизайн. Его популяризации в новое время послужили публикация листов Letraset с образцами Lorem Ipsum в 60-х годах и, в более недавнее время, программы электронной вёрстки типа Aldus PageMaker, в шаблонах которых используется Lorem Ipsum.
\begin{figure}[h]
    \centering
    \includegraphics[width=0.25\textwidth]{example-image-a}
    \caption{Правильно подпись делает, между прочим!}
    \label{fig:mesh1}
\end{figure} 
\label{sec:main}
\subsection{Подраздел}
\label{subsec:main}
\subsubsection{Подподраздел}
\label{subsubsec:main}
\section{очень-очень-очень-очень-очень-очень-очень-очень длинное название раздела (такое бывает)}
Как было сказано в разделе \ref{sec:main}...\\
Как было сказано в подразделе \ref{subsec:main}...\\
Как было сказано в подподразделе \ref{subsubsec:main}...\\
\addcontentsline{toc}{section}{СПИСОК ВИКОРИСТАНИХ ДЖЕРЕЛ} 
\section*{список використаних джерел}
Ваш список источников
\end{document}
