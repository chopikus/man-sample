\documentclass[14pt]{extarticle}
\usepackage{fontspec}
\setmainfont[Ligatures=TeX]{Times New Roman}
\usepackage[russian, ukrainian, english]{babel}
\usepackage{extsizes}
\usepackage{geometry}
 \geometry{
 a4paper,
 total={170mm,257mm},
 left=2cm,
 top=2cm,
 bottom=2cm,
 right=2cm	
 }
\def\changemargin#1#2{\list{}{\rightmargin#2\leftmargin#1}\item[]}
\let\endchangemargin=\endlist 

\begin{document}
\begin{titlepage}
\begin{center}
Міністерство освіти і науки України\\
Департамент науки і освіти Харківської облдержадміністрації\\
Харківське територіальне відділення МАН України\\
\end{center}
\par\null\par\null\par
\hfill
\begin{tabular}{@{}l@{}}
Відділення: комп'ютерних наук\\
Секція: комп'ютерні системи та
мережі
\end{tabular}
\par\null\par\null\par
\begin{center}
РОЗРОБКА МЕТОДА ВИПРАВЛЕННЯ, ЯК НЕЗАЛЕЖНИХ, ТАК І ГРУПОВИХ
ПОМИЛОК, ЩО ВИНИКАЮТЬ ПРИ ОБМІНІ ІНФОРМАЦІЄЮ В
КОМП'ЮТЕРНИХ МЕРЕЖАХ
\end{center}
\par\null\par\null\par\null
%\hfill
%\begin{tabular}{@{}l@{}}
\begin{changemargin}{10cm}{0cm}
Роботу виконав:\\
Волков Юрій Володимирович,\\
учень 10 класу Харківського\\
Навчально-виховного комплексу\\
№45 «Академічна гімназія»\\
Харківської міської ради\\
Харківської області
%\end{tabular}
\end{changemargin}
\par
\begin{changemargin}{10cm}{0cm}
%\hfill
%\begin{tabular}{@{}l@{}}
Науковий керівник:\\
Руккас Кирило Маркович,\\
доцент Харківського\\
національного університету\\
імені В.Н.Каразіна, кандидат\\
технічних наук
%\end{tabular}
\vspace*{\fill}
\end{changemargin}
\begin{center}
Харків -- 2014 
\end{center}
\end{titlepage}
САМЫЙ КРУТОЙ МАН НА СВЕТЕ!1!
\end{document}
